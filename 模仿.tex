% !Mode:: "TeX:UTF-8"
% !TEX program  = xelatex
% !BIB program = bibtex
%\documentclass{cumcmthesis}
\documentclass[withoutpreface,bwprint]{cumcmthesis} %去掉封面与编号页

\usepackage[framemethod=TikZ]{mdframed}
\usepackage{url}   % 网页链接
\usepackage{subcaption} % 子标题
\numberwithin{equation}{subsection}
\newcommand{\toper}{\noindent \textbf}
\renewcommand\theequation{\arabic{subsection}-\arabic{equation}}

\title{炉温曲线指标设定控制策略}


复合肥饿了绿萼TV热退热退热退热特瑞特让他如题如题日日日日日日日日日日日日日日日 她她她她她她她日日日她她她她她她她她她她她她她她她她她她她 日日日日日日日日日日日日日日日日日日日她她她她她她她她她她她她她她她她她她她让人
\begin{document}
的嘀咕嘀咕

 \maketitle
 \begin{abstract}
本文研究的是回焊炉的温度与传送带运送电路板速度的最佳策略问题,通过建立电子元件和回焊炉内空气的热传导模型,求解得到不同条件和目标下的最优炉温曲线。

\textbf{针对问题一:}

\textbf{针对问题二:}在问题一的机理模型基础上,在满足制程界限的约束下,以过炉速度最大为目标,以制程界限为约束条件,通过控制速度参数对模型使用\textbf{遍历算法},遍历得到速度$v$的最大值。最终求解得最大速度为:\textbf{78.26cm/min。}

\textbf{针对问题三:}由题意要求曲线中大于217℃的覆盖面积最小,曲线面积为时间对温度的积分。以该段函数面积最小为目标,满足制程界限为约束条件,建立优化模型。对每个小温区的温度和过炉速度进行\textbf{循环遍历求解},得到满足目标以及约束的最优温度设定及过炉速度,并求出该初始参数下的最优炉温曲线。得到设定温度为:小温度1~5:\textbf{169℃};小温区6: \textbf{180℃};小温区7:\textbf{240℃};小温区8~9:\textbf{216℃},传送带过炉速度为:\textbf{85cm/min}时,炉温曲线上超过217℃到峰值温度所覆盖的面积最小为:\textbf{429.73}。

\textbf{针对问题四:}本题以达到温度峰值的时间作为对称轴,以对称轴两边对称性最好、超过217℃所覆盖面积最小为目标,\textbf{建立双目标优化模型},对各小温区温度以及过炉速度进行\textbf{循环遍历求解},找到各初始参数时所求炉温曲线为最优解。最终得到\textbf{最优炉温曲线}的各指标值为:设定温度为:小温度1~5:\textbf{181℃};小温区6:\textbf{201℃};小温区7:\textbf{240℃};小温区8~9:。传送带过炉速度为:\textbf{97cm/min},所覆盖的面积为\textbf{477.18}。

\textbf{关键词}:热传导模型、优化模型、循环遍历求解、初始参数控制策略。
\end{abstract}





\section{问题的背景与重述}

\subsection{问题的提出}
小型电子元件的不断出现,对焊接技术提出了更高的要求。利用回焊炉进行加热向电路板上焊接电子元件时,回焊炉的温度设置对最终的焊接产品质量有密切影响。回焊炉的温度控制与调整过程现阶段可通过机理模型来设置。

回焊炉的加热过程可理解为:电路板放置在传送带上进入炉内,依次经过回焊炉中由11个小温区构成的4个功能区(预热、恒温、回流、冷却)进行焊接。要得到质量优秀的产品,需分析研究如何设定各个温区的温度和传送带的速度,得到最优情况下的炉温曲线。

\subsection{问题重述}
现已知某种回焊炉内部的具体构造,生产车间的具体环境,各小温区温度设定与调整的范围与要求,试分析各温区的设定温度、传送带传送速度以及满足制程界限的炉温曲线。

\begin{enumerate}[label=(\arabic*)]
\item  已知传送带过炉速度,建立焊接区域温度变化规律的数学模型,求出给定温度条件下的炉温曲线。
	
\item  已知各温区设定的温度,分析允许的最大传送带过炉速度。
	
\item  理想情况下,炉温曲线应使超过217℃到峰值温度的覆盖面积最小。求满足要求的最优炉温曲线;并给出此条件下的传送带速度与各温区的温度,求出覆盖面积。
	
\item 炉温曲线在满足制程界限下,还应使峰值温度两侧超过217℃的部分尽量对称。在此要求下,对问题3得到的结果进一步研究,求出改进后问题3的各项解及相应指标。
\end{enumerate}




\section{问题分析}
在回焊炉中进行焊接时,影响焊接质量的因素有:各温区的温度设定、传送带过炉速度。为了保证产品的生产要求,还应使得电路板在焊接过程中,焊接区域得炉温曲线满足制程界限。

\subsection{问题1的分析}
将4个大温区以温度分布类型为标准,细分为10个区域,对这些区域依次求解研究回焊炉内的温度分布。电路板外侧温度变化是由与回焊炉内空气温度热交换产生的。热交换主要有热传导,热对流和热辐射三种形式。本题中,主要体现为热传导的形式。因此,对电路板焊接区域建立三维热传导模型来求得稳定状态下的温度分布,并将三维热传导模型转化为一维热传导模型。

对焊接区域分段建立热传导模型,由于这些区域中的温度分布类型不同,因此,需对这些区域内部的空气温度进行分析或同样建立热传导方程,计算得到对焊接区域的边界条件。

利用附件数据,通过最小二乘法计算电路板热传导模型的参数$a^2$。逐步差分法对离散化后的热传导模型求解,探究给定温度与传送带温度的条件下,电路板的温度分布规律。具体求解流程如下:




\begin{figure}[!h]
    \centering
    \includegraphics[width=.85\textwidth]{program figure}
    \caption{问题一求解流程}
    \label{fig:program_figure}
\end{figure}


\subsection{问题2的分析}
问题二给出每个小温区的温度,需要求解满足制程界限的最大速度,首先找到$v$与时间的关系,通过时间与温度的关系建立目标函数,使电路板走过回焊炉的时间最短,时间与温度的关系的数值解由第一问求解得到。由题目给出的制程界限列出约束条件。将目标函数以及约束条件离散化,进行\textbf{遍历算法} 求解。由遍历算法求出满足条件的最优解。

\subsection{问题3的分析}
问题三既没有给出每个小温区的温度,也没有给出过炉速度,即需要我们对所有的小温区温度进行循环遍历求解。首先写出炉温曲线超过217\textcelsius 的面积表达式,通过曲线上面积与时间及温度的关系建立目标函数,在满足制程界限的前提下,通过遍历求解出使曲线超过217\textcelsius 的面积最小的各小温区最优温度设定以及最优过炉速度。

\subsection{问题4的分析}
本问在第三题的基础上,找到满足制程界限的、超过217\textcelsius 覆盖面积最小,且满足炉温曲线在217℃以上以峰值温度为中心线,两侧部分的曲线对称性最好的最优炉温曲线。

以炉温曲线满足覆盖面积最小、217\textcelsius 以上图像关于$x$ = $T_{c}$轴尽量对称为双目标,以各温区的温度$T$,传送带过炉速度$v$为决策变量的双目标规划模型;尽量对称的目标函数可表示为使曲线上到对称轴距离相等的点,温度差尽可能小。计算时利用问题三求得的结果将覆盖面积转化为约束,求解以对称为单目标的规划模型。
\section{模型假设}

\begin{enumerate}[label=(\arabic*)]
	\item
	假设电路板与传送带之间无相对位移。
	\item
	假设炉内温度在垂直于传送带的运动方向上,温度相等。
	\item
	假设小温区供热稳定

	
\end{enumerate}






\section{符号说明}


\begin{table}[h]
	\centering
	\begin{tabular}{ccc}
		\toprule[1.5pt]
		符号   &    表示含义 &   单位  \\
		\midrule[1pt]
		$u$ & 物体温度 & \textcelsius  \\
		$T_{n}$ & 回焊炉边界温度 & \textcelsius \\
		$x$& 沿传送带方向的横坐标 &  $cm$ \\
		$y$& 垂直于传送带方向的纵坐标 &  $cm$   \\
		$l$& 所焊接电路板的厚度 &   $cm$  \\
		$v$& 传送带的过炉速度 &   $cm/min$  \\
		$d$& 划定区域的长度 &    $cm$ \\
		$s$& 焊接区域中心走过长度范围 &   $cm$  \\
		$S$& 炉温曲线超过217℃到峰值温度的覆盖面积 &   $cm^2$  \\
		\bottomrule[1.5pt]
		
	\end{tabular}
\end{table}

\section{模型准备}
\subsection{附件数据处理}
\subsubsection{炉温曲线的绘制与温度变化分析}
 附件是某次实验在数值设定为表 \ref{speed_1}   的情况下,得到的炉温曲线数据:
\begin{table}[!htbp]
 	\caption[speed1]{附件数据温区温度与传送速度}
 	\label{speed_1}
 	\centering
 	\begin{tabular}{cccccc}
 		\toprule[2.0pt]
		温区 & 1 \~{} 5 & 6 & 7 & 8 \~{} 9 & 10 \~{} 11\\
 		\midrule[1pt]
 		温度 & 175\textcelsius & 195\textcelsius & 235\textcelsius & 255\textcelsius &  25\textcelsius \\
 		\midrule[1pt]
 		传送带速度 &\multicolumn{5}{c}{$70cm/min$}\\
 		\bottomrule[2.0pt]
 	\end{tabular}
\end{table}


对所给数据拟合(附录1)得到图\ref{time1}所示的炉温曲线图和图\ref{time2}所示的温度变化率与时间的关系。

\begin{figure}[!h]
	\centering
	\begin{minipage}[c]{0.48\textwidth}
		
		\centering
		\includegraphics[height=0.2\textheight]{time1}
		\subcaption{附件数据炉温曲线}
		\label{time1}
	\end{minipage}
	\begin{minipage}[c]{0.48\textwidth}
		
		\centering
		\includegraphics[height=0.2\textheight]{time2}
		\subcaption{温度变化率与时间关系}
		\label{time2}
	\end{minipage}
	\label{fig:temp_and_time}
\end{figure}

\subsubsection{大温区判断}

回焊炉的大温区划分标准为焊接过程中的温度变化情况。预热区与回流区温度上升较快;恒温区温度趋于稳定,温差小;冷却区温度降低。
将图3的数据与温区坐标结合分析,温度变化率的最小取值为$0.16\~{}0.28℃/s$,出现在传送带运动$138s$之后,$166.5s$附近,此时电路板处于第五个小温区。由此判断第五个小温区为恒温区。同理,小温区10、11为冷却区。综上,得到回焊炉内部的大温区依次为:

\begin{table}[H]
	 
	\centering
	\caption[temper1]{大温区对应的各个小温区}
	\label{temper1}
	
	\begin{tabular}{cccc}
		\specialrule{0em}{0pt}{0pt}
		\toprule[1.5pt]
		预热区   & 恒温区   &回流区 & 冷却区  \\
		\midrule[1pt]
		1\~{}4 & 5 & 6\~{}9 & 10、11\\
		\bottomrule[1.5pt]
	\end{tabular}
\end{table}

\subsection{温度区域分区处理}
对焊接区域作以下分区域处理,建立分段函数表示每个区域的位置坐标,划分的区域类型如下:

\begin{enumerate}
	\item \textbf{恒定温度区域,}
	
	(1)若相邻小温区温度相同,则这些小温区间隙温度相同。因此,可分出4个区域分别为:小温区1\~{}5及其间隙,8\~{}9及其间隙,小温区6和小温区7。
	\item \textbf{变温间隙},若相邻小温区的温度不同,则两者间隙中存在空气的热传导,且由题意“达到稳定”可知,这些间隙温度分布不再改变;分出4个区域。
	\item \textbf{冷却温区},温区9\~{}10、其间隙、炉后区无热源加热,分为1个区域。
	\item \textbf{炉前区域},
	
	炉前区域自身为1个区域。
	综上所述,我们将4个大温区及炉前炉后区域分为了10个小区域,每个区域建立单独的模型进行计算。区域划以及炉内稳定后空气温度分布分示意图如下:
	
	\begin{figure}[!h]
	\centering
	\includegraphics[width=.85\textwidth]{space temper}
	\caption{区域划以及炉内稳定后空气温度分布分示意图}
	\label{fig:temper}
	\end{figure}
	划分后的10个区域范围(与炉前区域起始位置的距离)为:
	
	\begin{table}[H]
		\caption{划分区域对应的位置范围}
		\label{tab:where} 
		\centering
		
		\begin{tabular}{c|c|c|c|c|c}
			\toprule[1.5pt]
			区域 & $1$ & $2$ &$ 3$ &$ 4$ &$ 5$ \\
			\hline
		    位置 & $0\~{}25 $& $25\~{}197.5 $& $197.5\~{}202.5$ &$ 202.5\~{}233$ & $233\~{}238$ \\
		    \hline
		    区域 & $6$ & $7$ &$ 8 $& $9$ & $10 $\\
		    \hline
		    位置 & $238\~{}268.5$ & $268.5\~{}273.5$ &$273.5\~{}339.5$ & $339.5\~{}344.5$ & $344.5\~{}435.5 $\\
			\bottomrule[1.5pt]	
		\end{tabular}
	\end{table}
	
\end{enumerate}

\subsection{坐标系的建立}
\subsubsection{以电子元件为参考的坐标系(下称坐标系1)}
以电路板底边中心为原点,以竖直向上为正方向,建立坐标轴$y$轴。因此,$y$轴是随电路板运动的坐标系。
\subsubsection{以回焊炉为参考的坐标系(下称坐标系2)}
为了分析电路板随传送带以速度$v$运动时,在每个区域上被加热或冷却的边界条件,以炉前区域出发点为原点,以沿传送运动方向为正方向,建立坐标轴$x$轴。
所建坐标系如图所示,在$𝑥$轴上确定区域及边界条件,在$y$轴上分析电路板的温度分布情况。



\begin{figure}[!h]
	\centering
	\includegraphics[width=.85\textwidth]{坐标}
	\caption{左图为对电路板建立坐标系 \quad   右图为对传送带建立坐标系}
	\label{fig:x_y}
\end{figure}

\section{模型的建立与求解}

\subsection{问题一}

\subsubsection{焊接区域温度随时间变化的模型建立}
电路板外侧温度变化是由与回焊炉内空气温度热交换产生的,而此过程可以由热传导进行描述,于是我们对电路板焊接区域建立三维热传导模型\cite{yangshimin2006}:

$$
\frac{\partial u}{\partial t}=a^{2}\left(\frac{\partial^{2} u}{\partial x^{2}}+\frac{\partial^{2} u}{\partial y^{2}}+\frac{\partial^{2} u}{\partial z^{2}}\right)
$$
其中:$a^2$为常系数,$a^2= \frac{K}{c\rho}$;

$u$表示温度的函数;

该模型表示电路板温度随时间和板内位置的变化情况。本题要求焊接区域温度变化规律模型。在三维空间中,由方程的连续性可得,沿电路板长、宽、高的任何一个方向计算几何中心点的温度相同。即从任意方向求解得到的几何中心温度相同。因此该题可化简为一维热传导问题,由于题中已给出厚度$𝑙=0.15𝑚$ ,因此对焊接区域建立坐标系$1$内的一维热传导方程\cite{yangshimin2006}:

\begin{equation}
	\frac{\partial u}{\partial t}=a^{2} \frac{\partial^{2} u}{\partial y^{2}}
	\label{6.0}
\end{equation}
\subsubsection{恒定温度区域的模型建立}

由图\ref{tab:where},恒定温度区域包含区域2、4、6、8;这些区域内部的温度可视为常量,电路板处于恒定温度区域时,可视为仅受恒温热源加热,建立一维热传导模型:

$$\frac{\partial u}{\partial t}= a^2\frac{\partial^2 u}{\partial y^2}$$


\noindent \textbf{边界条件:}

每个区域的边界条件为\cite{lishan2020}:

\begin{equation}
	\left\{\begin{array}{l}
		u\left(0, t_{n}\right)=u\left(l, t_{n}\right)=T_{n} \\
		t_{n}=t-\frac{x_{n}^{0}}{v}
	\end{array}\right.
	\label{6.2}
\end{equation}

\noindent 其中:$T_{n}$表示第$n (𝑛 =2,4,6,8)$个温度分布区域下,焊接区域的边界温度;

$t_{n}$表示在第$n$ 个区域运动的时间

$t$表示电路板运动总时间


$s_{n}$表示第$n$ 个区域的位置范围

$s_{n}^0$表示第$n$ 个区域的开始位置

\noindent \textbf{初始条件:}

每个区域的初始条件为:

\begin{equation}
	\begin{array}{c}
	u\left(y,t_{n}^0 \right) =T_{n}^0 \\
	t_{n}=\frac{s_{n}^0}{v}
	\end{array}
		\label{6.3}
\end{equation}

\noindent 其中$t_{n}^0$:表示焊接区域刚进入第$n$ 个区域的初始时刻;

$T_{n}^0$表示焊接中心进入第$n$ 个区域的初始温度,其值为上一个区域末状态的温度$T_{n-1}^\text{末}$。即:$T_{n}^0=T_{n-1}^\text{末}$

综上所述,结合公式\eqref{6.0}\~{}\eqref{6.3},建立恒定温度区域的数学模型\cite{liuhaiyang2013latex}:

$$\frac{\partial u}{\partial t}= a^2\frac{\partial^2 u}{\partial y^2}$$

\begin{equation}
	\left\{ \begin{array}{l}
			u\left(y,t_{n} \right) =u\left(l,t_{n} \right)=T_{n}\\
			u\left(y,t_{n}^0 \right) =T_{n}^0\\
			T_{n}^0= T_{n}^\text{末}\\
			n=2,4,6,8
		\end{array}\right.
	\label{6.4}
\end{equation}

\subsubsection{炉前区域的模型建立}

\noindent 炉前区域对电路板建立热传导模型

电路板在炉前区域即进入被加热状态,在厚度方向上的热传导模型同\eqref{6.4}:

$$\frac{\partial u}{\partial t}= a^2\frac{\partial^2 u}{\partial y^2}$$

\begin{equation}
	\left\{ \begin{array}{l}
		u\left(y,t_{n} \right) =u\left(l,t_{n} \right)=T_{n}\\
		u\left(y,t_{n}^0 \right) =T_{n}^0\\
		n=1
	\end{array}\right.
	\label{6.5}
\end{equation}

\noindent 其中:$T_{1}^0=25\textcelsius$ ,为刚进入区域\textcircled{1}时,焊接区域中心的温度。

$T_{1}$为$y$方向上焊接区域边界的温度,其值需对炉前区空气进行热传导分析。

\noindent \textbf{对炉前区空气建立热传导模型求解初始条件}

由图\ref{fig:temp_and_time},区域1为炉前区域,炉前区域最左侧温度为室温25\textcelsius,最右侧温度为区域\textcircled{2}的温度$T_{1}$。炉前区域的温度分布稳定,但不相同。对空气建立在坐标系2内的一维热传导模型:

\begin{equation}
	\frac{\partial U_{n}}{\partial t}=a^{2} \frac{\partial^{2} U_{n}}{\partial x^{2}}
	\label{6.6}
\end{equation}

\begin{equation}
	\left\{\begin{array}{l}
	U_{n}(0, t)=25^{\circ} \mathrm{C} \\
	U_{n}(25, t)=T_{1} \\
	U_{n}(x, 0)=25^{\circ} \mathrm{C}
	\end{array}\right.
	\label{6.7}
\end{equation}

由式\eqref{6.7}可以求得温度随$𝑥$轴坐标变化的函数()ux炉前,将其作为电路板在炉前区的边界条件。综合式\eqref{6.2}\~{}\eqref{6.7},建立炉前区域的温度变化模型:

$$
\frac{\partial u}{\partial t}=a^{2} \frac{\partial^{2} u}{\partial y^{2}}
$$

\begin{equation}
	\left\{ \begin{array}{l}
		u\left(y,t_{n} \right) =u\left(l,t_{n} \right)=U_{n}(x)\\
		u\left(y,t_{n}^0 \right) =T_{n}^0\\
		n=1
	\end{array}\right.
	\label{6.8}
\end{equation}

\subsubsection{变温间隙的模型建立}

\noindent \textbf{电路板边界条件的确定}

变温间隙包含区域\textcircled{3}、\textcircled{5}、\textcircled{7},其空气温度分布模型的建立与$u_{n}(x)$的求解与炉前区域相同,只需求得这几个区域的边界条件:
\begin{equation}
	\left\{\begin{array}{l}
		T_{n}=U_{n}(x)\\
		n=3,5,7
	\end{array}\right.
\end{equation}

\noindent \textbf{过间隙时焊接区域温度模型}

设$T_{n}$表示焊接区域进入第$𝑛$ 个区域的初始温度;$T_{n}^\text{末}$表示第$𝑛$ 各区域末边界的温度值,由于炉内空气会在短时间内达到稳定,即炉内空间某点的温度不再随时间发生变化,温度值连续。则第$𝑛 +1$个区域的初始值可表示为上一个区域末边界的温度:

\begin{equation}
	\left\{\begin{array}{l}
	T_{n+1}^{0}=T_{n}^{\text {末 }} \\
	T_{1}^{0}=25^{\circ} \mathrm{C}
	\end{array}\right.
\end{equation}

设$d_{n}$为第$𝑛$ 个区域的长度,建立的变温间隙的模型为:

$$
\frac{\partial u}{\partial t}=a^{2} \frac{\partial^{2} u}{\partial y^{2}}
$$
\begin{equation}
	\left\{\begin{array}{l}
	u\left(0, t_{n}\right)=u\left(l, t_{n}\right)=U_{n}(x) \\
	u\left(y, t_{n}^{0}\right)=T_{n}^{0} \\
	n=3,5,7
	\end{array}\right.
\end{equation}

\subsubsection{冷却区模型建立}
同先前模型,建立电路板的的数学模型:
$$
\frac{\partial u}{\partial t}=a^{2} \frac{\partial^{2} u}{\partial y^{2}}
$$
\begin{equation}
	\left\{\begin{array}{l}
	u\left(0, t_{n}\right)=u\left(l, t_{n}\right)=U_{n}(x) \\
	u\left(y, t_{n}^{0}\right)=T_{n}^{0} \\
	n=9,10
	\end{array}\right.
	\label{6.12}
\end{equation}

由于小温区10、11温度为25\textcelsius,与室温相同,因此可以近似看作:当焊接区域刚过小温区9时,视为开始无热源加热。小温区9末端到炉后区最末端距离为$96cm$,可以认为炉后区末端温度为25\textcelsius,而其温度变化与$y$方向无关,因此对小温区9末端到炉后区末端空气建立与炉前温度类似的一维热传导模型:

于是得到:

$$
\frac{\partial U_{n}}{\partial t}=a^{2} \frac{\partial^{2} U_{n}}{\partial x^{2}}
$$
\begin{equation}
	\left\{\begin{array}{l}
	U_{n}(339.5, t)=T_{n} \\
	U_{n}(435.5, t)=25^{\circ} \mathrm{C} \\
	U_{n}(x, 0)=T_{n} \\
	n=8
	\end{array}\right.
	\label{6.13}
\end{equation}

综合式\eqref{6.12}\~{}\eqref{6.13},建立炉后区域的温度变化模型:
$$
\frac{\partial u}{\partial t}=a^{2} \frac{\partial^{2} u}{\partial y^{2}}
$$
\begin{equation}
	\left\{\begin{array}{l}
	u\left(0, t_{n}\right)=u\left(l, t_{n}\right)=U_{n}(x) \\
	u\left(y, t_{n}^{0}\right)=T_{n}^{0} \\
	n=9,10
	\end{array}\right.
	\label{6.14}
\end{equation}

综上所述,总模型如下:

$$
\frac{\partial u}{\partial t}=a^{2} \frac{\partial^{2} u}{\partial y^{2}}
$$

$$
\frac{\partial U_{n}}{\partial t}=a^{2} \frac{\partial^{2} U_{n}}{\partial x^{2}}
$$



\begin{equation}
	\left\{\begin{array}{l}
		\left\{\begin{array}{l}
		u\left(0, t_{n}\right)=u\left(l, t_{n}\right)=U_{n}(x) \\
		u\left(y, t_{n}^{0}\right)=T_{n}^{0} \\
		T_{n}^{0}=T_{n-1}^{\text {末}}
		\end{array}\right. \\
		\left\{\begin{array}{l}
		U_{n}\left(s_{n}^{0}, t\right)=T_{n} \\
		U_{n}\left(s_{n}^{\text {末}}, t\right)=25^{\circ} \mathrm{C} \\
		U_{n}(x, 0)=T_{n}
		\end{array} \quad n=1,2, \ldots, 10\right.\\
		t_{n}^{0}=\frac{s_{n}^{0}}{v} \\
		t_{n}=t-\frac{s_{n}^{0}}{v}
	\end{array}\right.
\end{equation}


\subsubsection{有限差分法\cite{lishan2020}进行模型求解}

\begin{enumerate}
	\item \textbf{步长确定}
	
	取时间步长$ℎ=0.5$,空间步长$p$。将电路板沿厚度方向分为$N_{0}$个小区间,电路板通过每个区域的所用的时间$time_{n}$分为$N_{n}$个小区间,由式\eqref{6.2}\~{}\eqref{6.3}得到电路板在每个区域的运动时间为:
	\begin{equation}
		\text { time }_{n}=\frac{s_{n}^\text{末}-s_{n}^{0}}{v},(n=1,2, \ldots, 10)
	\end{equation}

	所以,对10个小区域依次差分处理:

	$$\left\{\begin{array}{l}
	y_{i}=i p,\left(i=1,2, \ldots, N_{0}\right) \\
	\text { time }_{n}=j h,\left(i=1,2, \ldots, N_{n}\right)
	\end{array}\right.$$
	由此固定网格划分。
	\item \textbf{将模型离散化}
	
	设将温度离散化后,将$u$分为横向$𝑖$ 个,纵向$j$层个微元为$u_{i}^j$,得到\cite{lishan2020}:

	\begin{equation}
		\begin{aligned}
		&\frac{\partial u}{\partial t}=\frac{u_{i}^{j+1}-u_{i}^{j}}{h} \\
		&\frac{\partial u}{\partial y}=\frac{u_{i}^{j+1}-u_{i-1}^{j}+u_{i+1}^{j+1}-u_{i-1}^{j+1}}{4 p} \\
		&\frac{\partial^{2} u}{\partial y^{2}}=\frac{u_{i+1}^{j}-2 u_{i}^{j}+u_{i-1}^{j}+u_{i+1}^{j+1}-2 u_{i}^{j+1}+u_{i-1}^{j+1}}{2 p^{2}}
		\end{aligned}
		\label{6.17}
	\end{equation}
	边界条件为:
	$$\begin{gathered}
	u(0, j h)=u(l, j h)=T_{n} \\
	u\left(i p, t_{n}^{0}\right)=T_{n}^{0}
	\end{gathered}$$

	\item 差分处理
	对式\eqref{6.1}进行差分得到\cite{lishan2020}:

	\begin{equation}
		\frac{u_{i}^{j+1}-u_{i}^{j}}{h}=a^{2} \frac{u_{i+1}^{j}-2 u_{i}^{j}+u_{i-1}^{j}+u_{i+1}^{j+1}-2 u_{i}^{j+1}+u_{i-1}^{j+1}}{2 p^{2}}
		\label{6.18.5}
	\end{equation}

	于是得到:
		$$
	u_{i}^{j+1}=k\left(u_{i+1}^{j}-2 u_{i}^{j}+u_{i-1}^{j}+u_{i+1}^{j+1}-2 u_{i}^{j+1}+u_{i-1}^{j+1}\right)+u_{i}^{j}
	$$
	\begin{equation}
		\left\{\begin{array}{l}
		u(0, j h)=u(l, j h)=T_{n} \\
		u\left(i p, t_{n}^{0}\right)=T_{n}^{0} \\
		i=1,2, \ldots, \frac{l}{p} ; j=1,2, \ldots, \frac{t}{h} \\
		n=1,2, \ldots, 10
		\end{array}\right.
		\label{6.18}
	\end{equation}

	其中,$k=\frac{a^2h}{2p^2}$

\end{enumerate}

\subsubsection{最小二乘法确定模型系数$a^2$}

建立热传导模型表示任意进入回焊炉的电路板温度随时间的变化情况,式中的$a^2$是一个给定温度下的确定参数,附件中的一组实验数据也适用于该模型。为了求解模型,需要确定参数$a$与各区域的边界条件。因此,利用附件给出的数据确定任意温度下参数$a$的值。

对厚度为$l$,焊接时长为$t$的焊接电路板,对\eqref{6.18}化简得:

\begin{equation}
	u_{i}^{j+1}=k\left(u_{i+1}^{j}-2 u_{i}^{j}+u_{i-1}^{j}+u_{i+1}^{j+1}-2 u_{i}^{j+1}+u_{i-1}^{j+1}\right)+u_{i}^{j}
	\label{6.20}
\end{equation}

其中,$k=\frac{a^2h}{2p^2}$,$k$是一个随温度改变的常数,由上式分析,当$k<0.5$时收敛,利用Matlab逐步差分法对$k\in[0.1,0.5]$进行逐次差分迭代(程序见附录3)求得该模型在任意时刻的理想温度值$U_{j}$,

利用最小二乘法进行拟合计算随温度改变的$k$值的实际值:
\begin{equation}
	min\sum_{i,j}(U_{i}^j-u_{j})
\end{equation}

步骤如下:

\textbf{step1}\quad 设时间步长$h=0.01$,空间步长$p=0.1$,$k$值步长为$q=0.01$,$n =1$;

\textbf{step2}\quad 预测值$k_{\beta}=0.1+\beta q,(\beta=1,2,...,40)$;

\textbf{step3}\quad 对第$i$个预测值,由公式计算温度预测值$U_{j}^i$,$U_{j}^i$表示由第$i$ 个$k$确定的各区域的温度预测值;

\textbf{step4}\quad 计算$k_{\beta}$对应的$\mu _{j}\sum_{i,j}(U_{i}^j-u_{j}),j=1,2,...,N_{n}$,并储存$\mu _{j}$值;

\textbf{step5}\quad 从储存的$\mu$值中选取最小值$\mu_{max}$,其对应$k$值即为第$n$ 个区域所求$k$;

\textbf{step6}\quad 若$n =10$,进行step7;否则,$n =n +1$并返回step2;

\textbf{step7}\quad 输出10个区域的$k$值。

\subsubsection{模型求解结果与分析}

\noindent \textbf{1. 结果展示}

最小二乘法求解得到求得$n (n =1,2,…,10)$个温度分布区域的每一段$k$值如下表所示(程序见附录3):


\begin{table}[!htbp]
    \label{tab:result1} 
	\centering
    \begin{tabular}{c|c|c|c|c|c|c|c|c|c|c}
        \toprule[1.5pt]
        区域&\textcircled{1}&\textcircled{2}&\textcircled{3}&\textcircled{4}&\textcircled{5}&\textcircled{6}&\textcircled{7}&\textcircled{8}&\textcircled{9}&\textcircled{10}\\
        \hline
        $k$值&0.476&0.5&0.49&0.5&0.5&0.5&0.5&0.5&0.296&0.192\\
		\hline
		拟合误差&0.04&0.18&0.02&0.07&0.01&0.01&0.10&0.06&0.07&0.03\\
        \bottomrule[1.5pt]
    \end{tabular}
\end{table}

将$k$值代入式\eqref{6.18.5}用有限差分法进行求解(程序见附录4)得到炉温曲线如下图所示:

\begin{figure}[!h]
	\centering
	\includegraphics{temper of the furnace}
	\caption{问题一的拟合炉温曲线求}
	\label{fig:furnace}
\end{figure}

求解得各点温度如下表所示:

\begin{table}[!htbp]
    \caption{各小温区温度}
	\label{tab:EOTT} 
	\centering
	
    \begin{tabular}{c|c|c|c|c}
        \toprule[1.5pt]
        所求点&小温区3中心点&小温区6中心点&小温区7中心点&小温区8末端点\\
        \hline
        温度& 129.82\textcelsius&167.78\textcelsius &184.53\textcelsius&209.41\textcelsius\\
        \bottomrule[1.5pt]
    \end{tabular}
\end{table}

\toper{2. 结果分析}

\toper{分析结果是否满足制程界限:}

本题所求温度曲线中各限制结果与制程界限对比如下表所示:


\begin{table}[!h]
    \caption{第一问温度曲线中各限制结果与制程界限对比}
	\label{tab:compete} 
	\centering
	
    \begin{tabular}{c|c|c|c|c}
        \toprule[1.5pt]
        界限&斜率&150℃~190℃时间&大于217℃时间&温度峰值\\
        \hline
        制程界限&$-3$\~{}$3$&$60$\~{}$120s$&$40$\~{}$90s$&$240$\~{}$250$℃\\
		\hline
		所求曲线&-$1.875\~{}2.3262$&$84s$&$89s$&$240.43℃$\\
        \bottomrule[1.5pt]
    \end{tabular}
\end{table}

由上表数据,本题求解结果均满足制程界限,且对图像的温度变化率分析,变化趋势与附件数据基本一致,为了判断数据的合理性,利用上述模型对附件数据进行温度拟合,并与真实值比较。

\toper{误差分析:}

将实验数据当作条件代入进行求解,得到实验数据的炉温曲线的理论值,将其与实际值进行对比分析:

\begin{figure}[!h]
	\centering
	\includegraphics{温度曲线}
	\caption{拟合炉温曲线与原曲线对比图}
	\label{fig:quxian}
\end{figure}

由图可以看出,两条曲线拟合效果较好,部分波段基本重合,但在炉前区与回流区的曲线差值较明显。对拟合炉温曲线与原曲线对比图进行误差分析,得到误差标准差为$1.18325$,其值较小,因此仍可以认为两条曲线拟合程度较高,可以认为本模型求解的理论值对实际值的预测较为合理。
\subsection{\quad 第二题}

\subsubsection{焊接中心温度随速度变化模型建立}

由问题一模型,将初始参数代入求解每一个离散后每时刻下焊接区域中心的温度值$u_{i}^j(\frac{1}{2},t_{j})$。本题要搜索的参数为速度,因此将速度作为求解目标,使用遍历算法进行求解。

首先建立目标函数即约束条件:

设焊接区域中心从起始点一直运动到终点所用的总时长$t_{\text{总}}$,则速度为:

其中,$s$为传送带最左端起始点到最右端终点的长度,$t$总为电路板在炉内运动的总时长。即若要求速度最大值,只需要求满足条件的$t$总最小值。

\subsubsection{最大速度控制模型建立}

\toper{1. 决策变量与目标函数}

\toper{决策变量:}需求出满足制程界限的传送带过炉速度$v$

\toper{目标函数:}满足制程界限的最大速度:

\begin{equation}
	\bm{maxv}
\end{equation}

其中:$v=\frac{s}{t_{\text{总}}}$

由此可见要求速度最大值,只需要求满足条件的$t_{\text{总}}$总最小值。

\toper{2. 约束条件}

\noindent (1) 温度变化斜率为:$\frac{du}{dt}$,要使温度上升斜率不超过$3$,下降斜率不低于$-3$:

$$-3\leq \frac{du}{dt}\leq 3$$

\noindent (2) 设电路板运动到坐标$x$处,温度为$u$时的时间为$t_{u}^{x}$,使温度上升时温度在$150℃$\~{}$190℃$区间内的时间不低于$60s$且不超过$120s$,:

$$60s\leq t^{150}-t^{190}\leq 120s$$

\noindent(3) 设温度从初始值升高到$217$\textcelsius 的时刻为$t_{a}^{217}$,从初始值升高到峰值温度$u_{max}$ 的时刻为$t_{c}^{u}$,从峰值温度下降到$217$\textcelsius 的时刻为$t_{b}^{217}$,要使温度大于$217$\textcelsius 的时长不低于$40𝑠$且不超过$90s$:

$$40s\leq t_{b}^{217}-t_{a}^{217}\leq 90s$$

\noindent (4) 要使峰值温度不低于$240$\textcelsius ,不高于$290$\textcelsius :

$$
240^{\circ} \mathrm{C} \leq u_{\max }(t) \leq 290^{\circ} \mathrm{C}
$$

\toper{综上所述,得到速度控制模型为:}
$$maxv$$


\begin{equation}
	\text { s.t. }\left\{\begin{array}{l}
		-3 \leq \frac{d u}{d t} \leq 3 \\
		60 \leq t^{190}-t^{150} \leq 120 \\
		240 \leq u_{\max }(t) \leq 290 \\
		40 \leq t_{b}^{217}-t_{a}^{217} \leq 90 \\
		v=\frac{s}{t_{b}}
	\end{array}\right.
\end{equation}

\subsubsection{模型求解}

\toper{模型离散化}

因u(t)为$u$随时间变化的解析解,无法进行求解,因此将模型离散化,用有限差分法进行数值解求解。


取时间步长为$h=0.5s$,将$t$分为$N=\frac{t_{\text{总}}}{h}$个小区间,则得到:

$$t_{j}=jh(j=1,2,...,N)$$

将温度离散化,将$u(t)$分为$j$个小微元$u_{j}$,得到:

$$\frac{du}{dt}=\frac{u_{j+1}-u_{j}}{h}$$

于是需满足:

$$-3\leq \frac{u_{j+1}-u_{j}}{h}\leq 3$$

由问题一的模型即方法可求解得到每个 $u_{j}$的数值,则设$j=j_{0}$时对应的温度为$u_{j_{0}}$,对应时间为$t_{j_{0}}$,而$t_{j_{0}}$为温度由初始值升高到150\textcelsius 的时刻。同理设$j= j_{\text{N}}$时
对应的温度为$u_{j_{\text{N}}}$,对应时间为$t_{j_{\text{N}}}$,而$t_{j_{\text{N}}}$为温度由初始值升高到190℃的时刻,于是得:

$$
60 \leq t_{j}^{190}-t_{j}^{150} \leq 120
$$

同理可得,温度大于217℃的时间约束为:

$$
\begin{aligned}
	&40 \leq t_{j b}^{217}-t_{j a}^{217} \leq 90 \\
	&j=1,2, \ldots, N
\end{aligned}
$$

使峰值温度不低于240℃,不高于290℃:

$$
240^{\circ} \mathrm{C} \leq u_{j c} \leq 290^{\circ} \mathrm{C}
$$

\toper {综上所述,得到速度控制离散模型为:}

$$maxv$$

\begin{equation}
	s.t.\left\{\begin{array}{l}
			-3\leq \frac{u_{j+1}-u_{j}}{h}\leq 3 \\
			60\leq t_{j}^{190} - t_{j}^{150} \leq 120 \\
			40\leq t_{jb}^{217}-t_{ja}^{217} \leq 90\\
			240^{\circ}C \leq u_{jc} \leq 290^{\circ}C\\
			v=\frac{s}{Nh}\\
			j = 1, 2,...,N \end{array}	\right.
\end{equation}

\subsubsection{遍历算法求解$v$}

本题用遍历算法寻找满足条件的最优速度值,遍历算法步骤如下:

step1\quad 初始值 $v=65$,设置速度变化步长0.01;计算此温度下的初始条件;

step2\quad  计算炉温曲线, 并判断其是否满足支撑界限。满足,保存$v$值,不满足,进行step3;

step3\quad $v=v+0.01$;判断$v$的值是否在(65,100)内, 满足,返回step1;否则,输出储存$v$的最大值。

\noindent 流程图如下:

\begin{figure}[!h]
	\centering
	\includegraphics{FlowChart}
	\caption{遍历算法流程图}
	\label{fig 8}
\end{figure}

\subsubsection{结果展示及分析}

由遍历算法求解(附录5)得到,当$v=78.26cm/min$,为满足制程界限的最大速度。将题目所给各小温区温度以及所求得最优速度作为初始参数代入Matlab求解((图程序见附录6)得到满足过炉速度最大的时侯的最优炉温曲线如下图所示:

\begin{figure}[!h]
	\centering
	\includegraphics{结果}
	\caption{问题二炉温曲线}
	\label{fig 9}
\end{figure}

\toper {结果分析:}

本题所求温度曲线中各限制结果与制程界限对比如下表所示:









\verb|\label{fig:circuit-diagram}| 是一个标签,供交叉引用使用的。例如引用图片 \verb|\cref{fig:circuit-diagram}| 的实际效果是\cref{fig:circuit-diagram}。图片是自动编号的,比起手动编号,它更加高效。\verb|\cref{label}| 由 \verb|cleveref| 宏包提供,比普通的 \verb|\ref{label}| 更加自动化。 \verb|label| 要确保唯一,命名方式推荐用图片的命名方式。

图片并排的需求解决方式多种多样,下面用 \verb|minipage| 环境来展示一个简单的例子。注意,以下例子用到了 \verb|subcaption| 命令,需要加载 subcaption 宏包。


这相当于整体是一张大图片,大图片引用是\cref{fig:sample-figure},子图引用别分是\cref{fig:sample-figure-a}、\cref{fig:sample-figure-b}、\cref{fig:sample-figure-c}。

如果原本两张图片的高度不同,但是希望它们缩放后等高的排在同一行,参考这个例子:


\section{绘制普通三线表格}
表格应具有三线表格式,因此常用 booktabs宏包,其标准格式如\cref{tab:001}~所示。
\begin{table}[!htbp]
    \caption{标准三线表格}\label{tab:001} \centering
    \begin{tabular}{ccccc}
        \toprule[1.5pt]
        $D$(in) & $P_u$(lbs) & $u_u$(in) & $\beta$ & $G_f$(psi.in)\\
        \midrule[1pt]
        5 & 269.8 & 0.000674 & 1.79 & 0.04089\\
        10 & 421.0 & 0.001035 & 3.59 & 0.04089\\
        20 & 640.2 & 0.001565 & 7.18 & 0.04089\\
        \bottomrule[1.5pt]
    \end{tabular}
\end{table}

其绘制表格的代码及其说明如下。
\begin{tcode}
    \begin{table}[!htbp]
        \caption[标签名]{中文标题}
        \begin{tabular}{cc...c}
            \toprule[1.5pt]
            表头第1个格   & 表头第2个格   & ... & 表头第n个格  \\
            \midrule[1pt]
            表中数据(1,1) & 表中数据(1,2) & ... & 表中数据(1,n)\\
            表中数据(2,1) & 表中数据(2,2) & ... & 表中数据(2,n)\\
            ...................................................\\
            表中数据(m,1) & 表中数据(m,2) & ... & 表中数据(m,n)\\
            \bottomrule[1.5pt]
        \end{tabular}
    \end{table}
\end{tcode}

\bigskip

table环境是一个将表格嵌入文本的浮动环境。tabular环境的必选参数由每列对应一个格式字符所组成:c表示居中,l表示左对齐,r表示右对齐,其总个数应与表的列数相同。此外,\verb|@{文本}|可以出现在任意两个上述的列格式之间,其中的文本将被插入每一行的同一位置。表格的各行以\verb|\\|分隔,同一行的各列则以\&分隔。 \verb|\toprule| 、\verb|\midrule| 和 \verb|\bottomrule| 三个命令是由booktabs宏包提供的,其中 \verb|\toprule| 和 \verb|\bottomrule| 分别用来绘制表格的第一条(表格最顶部)和第三条(表格最底部)水平线, \verb|\midrule| 用来绘制第二条(表头之下)水平线,且第一条和第三条水平线的线宽为 1.5pt ,第二条水平线的线宽为 1pt 。引用方法与图片的相同。

\section{公式}

数学建模必然涉及不少数学公式的使用。下面简单介绍一个可能用得上的数学环境。

首先是行内公式,例如 $ \theta $ 是角度。行内公式使用 \verb|$  $| 包裹。

行间公式不需要编号的可以使用 \verb|\[  \]| 包裹,例如
\[
E=mc^2
\]
其中 $ E $ 是能量,$ m $ 是质量,$ c $ 是光速。

如果希望某个公式带编号,并且在后文中引用可以参考下面的写法:
\begin{equation}
E=mc^2
\label{eq:energy}
\end{equation}
式\cref{eq:energy}是质能方程。

多行公式有时候希望能够在特定的位置对齐,以下是其中一种处理方法。
\begin{align}
P & = UI \\
& = I^2R
\end{align}
\verb|&| 是对齐的位置, \verb|&| 可以有多个,但是每行的个数要相同。

矩阵的输入也不难。
\[
\mathbf{X} = \left(
    \begin{array}{cccc}
    x_{11} & x_{12} & \ldots & x_{1n}\\
    x_{21} & x_{22} & \ldots & x_{2n}\\
    \vdots & \vdots & \ddots & \vdots\\
    x_{n1} & x_{n2} & \ldots & x_{nn}\\
    \end{array} \right)
\]

分段函数这些可以用 \verb|case| 环境,但是它要放在数学环境里面。
\[
f(x) =
    \begin{cases}
        0 &  x \text{为无理数} ,\\
        1 &  x \text{为有理数} .
    \end{cases}
\]
在数学环境里面,字体用的是数学字体,一般与正文字体不同。假如要公式里面有个别文字,则需要把这部分放在 \verb|text| 环境里面,即 \verb|\text{文本环境}| 。

公式中个别需要加粗的字母可以用 \verb|$\bm{math symbol}$| 。如 $ \alpha a\bm{\alpha a} $ 。

以上仅简单介绍了基础的使用,对于更复杂的需求,可以阅读相关的宏包手册,如 \href{http://texdoc.net/texmf-dist/doc/latex/amsmath/amsldoc.pdf}{amsmath}。

希腊字母这些如果不熟悉,可以去查找符号文件 \href{http://mirrors.ctan.org/info/symbols/comprehensive/symbols-a4.pdf}{symbols-a4.pdf} ,也可以去 \href{http://detexify.kirelabs.org/classify.html}{detexify} 网站手写识别。另外还有数学公式识别软件 \href{https://mathpix.com/}{mathpix} 。

下面简单介绍一下定理、证明等环境的使用。
\begin{definition}
    定义环境
    \label{def:nosense}
\end{definition}
\cref{def:nosense}除了告诉你怎么使用这个环境以外,没有什么其它的意义。

除了 definition 环境,还可以使用 theorem 、lemma、corollary、assumption、conjecture、axiom、principle、problem、example、proof、solution 这些环境,根据论文的实际需求合理使用。

\begin{theorem}
    这是一个定理。
    \label{thm:example}
\end{theorem}
由\cref{thm:example}我们知道了定理环境的使用。

\begin{lemma}
    这是一个引理。
    \label{lem:example}
\end{lemma}
由\cref{lem:example}我们知道了引理环境的使用。

\begin{corollary}
    这是一个推论。
    \label{cor:example}
\end{corollary}
由\cref{cor:example}我们知道了推论环境的使用。

\begin{assumption}
    这是一个假设。
    \label{asu:example}
\end{assumption}
由\cref{asu:example}我们知道了假设环境的使用。

\begin{conjecture}
    这是一个猜想。
    \label{con:example}
\end{conjecture}
由\cref{con:example}我们知道了猜想环境的使用。

\begin{axiom}
    这是一个公理。
    \label{axi:example}
\end{axiom}
由\cref{axi:example}我们知道了公理环境的使用。

\begin{principle}
    这是一个定律。
    \label{pri:example}
\end{principle}
由\cref{pri:example}我们知道了定律环境的使用。

\begin{problem}
    这是一个问题。
    \label{pro:example}
\end{problem}
由\cref{pro:example}我们知道了问题环境的使用。

\begin{example}
    这是一个例子。
    \label{exa:example}
\end{example}
由\cref{exa:example}我们知道了例子环境的使用。

\begin{proof}
    这是一个证明。
    \label{prf:example}
\end{proof}
由\cref{prf:example}我们知道了证明环境的使用。

\begin{solution}
    这是一个解。
    \label{sol:example}
\end{solution}
由\cref{sol:example}我们知道了解环境的使用。



\section{其它小功能}
\subsection{脚注}

利用 \verb|\footnote{具体内容}| 可以生成脚注\footnote{脚注可以补充说明一些东西}。

\subsection{部分特殊符号的输入}

每门编程语言都会把部分符号拿来使用,要输出符号本身就需要转义, \TeX{}也不例外。

\begin{table}[!ht]
	\centering
	\begin{tabular}{cc}
		\toprule
		符号            &  输入                   \\
		\midrule
		\%              &  \verb|\%|             \\
		\&              &  \verb|\&|             \\
		\$              &  \verb|\$|             \\
		\{              &  \verb|\{|             \\
		\}              &  \verb|\}|             \\
		\#              &  \verb|\#|             \\
		\_              &  \verb|\_|             \\
		\^{}            &  \verb|\^{}|           \\
		\~{}            &  \verb|\~{}|           \\
		\textbackslash  &  \verb|\textbackslash| \\
		\bottomrule
	\end{tabular}
\end{table}

\subsection{无序列表与有序列表}

无序列表是这样的:
\begin{itemize}
    \item one
    \item two
    \item ...
\end{itemize}

有序列表是这样子的:
\begin{enumerate}
    \item one
    \item two
    \item ...
\end{enumerate}

\subsection{字体加粗与斜体}

如果想强调部分内容,可以使用加粗的手段来实现。加粗字体可以用 \verb|\textbf{加粗}| 来实现。例如: \textbf{这是加粗的字体。 This is bold fonts} 。

中文字体没有斜体设计,但是英文字体有。\textit{斜体 Italics}。

\section{参考文献与引用}

参考文献对于一篇正式的论文来说是必不可少的,在建模中重要的参考文献当然应该列出。\LaTeX{}在这方面的功能也是十分强大的,下面进介绍一个比较简单的参考文献制作方法。有兴趣的可以学习 \verb|bibtex| 或 \verb|biblatex| 的使用。

\LaTeX{}的入门书籍可以看《\LaTeX{}入门》\cite{liuhaiyang2013latex}。这是一个简单的引用,用 \verb|\cite{bibkey}| 来完成。要引用成功,当然要维护好 bibitem 了。下面是个简单的例子。

\newpage

%参考文献
\begin{thebibliography}{9}%宽度9
    \bibitem{yangshimin2006}杨世铭, 陶文铨. 传热学(第4版) [Heat Transfer][M]. 高等教育出版社, 2006.
    \bibitem{}李杉. 不同边界条件下热方程的解法与应用 [D]. 山东大学, 2020.
\end{thebibliography}

\newpage
%附录
\begin{appendices}

\section{模板所用的宏包}
\begin{table}[htbp]
    \centering
    \caption{宏包罗列}
    \begin{tabular}{ccccc}
        \toprule
        \multicolumn{5}{c}{模板中已经加载的宏包} \\
        \midrule
        amsbsy & amsfonts & {amsgen} & {amsmath} & {amsopn} \\
        amssymb & amstext & {appendix} & {array} & {atbegshi} \\
        atveryend & auxhook & {bigdelim} & {bigintcalc} & {bigstrut} \\
        bitset & bm    & {booktabs} & {calc} & {caption} \\
        caption3 & CJKfntef & {cprotect} & {ctex} & {ctexhook} \\
        ctexpatch & enumitem & {etexcmds} & {etoolbox} & {everysel} \\
        expl3 & fix-cm & {fontenc} & {fontspec} & {fontspec-xetex} \\
        geometry & gettitlestring & {graphics} & {graphicx} & {hobsub} \\
        hobsub-generic & hobsub-hyperref & {hopatch} & {hxetex} & {hycolor} \\
        hyperref & ifluatex & {ifpdf} & {ifthen} & {ifvtex} \\
        ifxetex & indentfirst & {infwarerr} & {intcalc} & {keyval} \\
        kvdefinekeys & kvoptions & {kvsetkeys} & {l3keys2e} & {letltxmacro} \\
        listings & longtable & {lstmisc} & {ltcaption} & {ltxcmds} \\
        multirow & nameref & {pdfescape} & {pdftexcmds} & {refcount} \\
        rerunfilecheck & stringenc & {suffix} & {titletoc} & {tocloft} \\
        trig  & ulem  & {uniquecounter} & {url} & {xcolor} \\
        xcolor-patch & xeCJK & {xeCJKfntef} & {xeCJK-listings} & {xparse} \\
        xtemplate & zhnumber &       &       &  \\
        \bottomrule
    \end{tabular}%
    \label{tab:addlabel}%
\end{table}%

以上宏包都已经加载过了,不要重复加载它们。

\section{排队算法--matlab 源程序}

\begin{lstlisting}[language=matlab]
kk=2;[mdd,ndd]=size(dd);
while ~isempty(V)
[tmpd,j]=min(W(i,V));tmpj=V(j);
for k=2:ndd
[tmp1,jj]=min(dd(1,k)+W(dd(2,k),V));
tmp2=V(jj);tt(k-1,:)=[tmp1,tmp2,jj];
end
tmp=[tmpd,tmpj,j;tt];[tmp3,tmp4]=min(tmp(:,1));
if tmp3==tmpd, ss(1:2,kk)=[i;tmp(tmp4,2)];
else,tmp5=find(ss(:,tmp4)~=0);tmp6=length(tmp5);
if dd(2,tmp4)==ss(tmp6,tmp4)
ss(1:tmp6+1,kk)=[ss(tmp5,tmp4);tmp(tmp4,2)];
else, ss(1:3,kk)=[i;dd(2,tmp4);tmp(tmp4,2)];
end;end
dd=[dd,[tmp3;tmp(tmp4,2)]];V(tmp(tmp4,3))=[];
[mdd,ndd]=size(dd);kk=kk+1;
end; S=ss; D=dd(1,:);
 \end{lstlisting}

 \section{规划解决程序--lingo源代码}

\begin{lstlisting}[language=c]
kk=2;
[mdd,ndd]=size(dd);
while ~isempty(V)
    [tmpd,j]=min(W(i,V));tmpj=V(j);
for k=2:ndd
    [tmp1,jj]=min(dd(1,k)+W(dd(2,k),V));
    tmp2=V(jj);tt(k-1,:)=[tmp1,tmp2,jj];
end
    tmp=[tmpd,tmpj,j;tt];[tmp3,tmp4]=min(tmp(:,1));
if tmp3==tmpd, ss(1:2,kk)=[i;tmp(tmp4,2)];
else,tmp5=find(ss(:,tmp4)~=0);tmp6=length(tmp5);
if dd(2,tmp4)==ss(tmp6,tmp4)
    ss(1:tmp6+1,kk)=[ss(tmp5,tmp4);tmp(tmp4,2)];
else, ss(1:3,kk)=[i;dd(2,tmp4);tmp(tmp4,2)];
end;
end
    dd=[dd,[tmp3;tmp(tmp4,2)]];V(tmp(tmp4,3))=[];
    [mdd,ndd]=size(dd);
    kk=kk+1;
end;
S=ss;
D=dd(1,:);
 \end{lstlisting}
\end{appendices}


\end{document} 